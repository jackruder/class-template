\section{Problem 2}

\claim{
	An array of integers $A[1...n]$ can be sorted in $O(n+D)$ time, where
	$D = \max_i A[i] - \min_i A[i]$ (allowing for use of extra storage).
}

\newcommand{\Nbox}[1]{
	\BeginBox[draw=blue,dashed,very thin]
	#1
	\BoxedString[draw=blue,very thin]{\color{blue}{\(O(n)\)}}
	\EndBox
}
\newcommand{\Dbox}[1]{
	\BeginBox[draw=red,dashed,very thin]
	#1
	\BoxedString[draw=red,very thin]{\color{red}{\(O(D+n)\)}}
	\EndBox
}

\pf{
First, see the following algorithm. The key observation
is that there are at most \(D + 1\) distinct values in \(A\).
This allows us to count the number of occurrences of each value,
and then fill in the sorted array by iterating over the counts.
\begin{algorithm}
	\caption{Sort an array of integers in \(O(n + D)\) time.}
	\begin{algorithmic}[1]
		\Require{Array \(A[1\ldots n]\), \(D = \max_i A[i] - \min_i A[i]\), \(A[i] \in \mathbb{Z}\)}
		\Ensure{Sorted array \(A[1\ldots n]\)}
		\Nbox{
			\State Find \(a_{min} = \min_i A[i]\) and \(a_{max} = \max_i A[i]\).
		}
		\LComment{Count the number of occurrences of each value in \(A\)}
		\Dbox{
			\State Initialize \(C[1\ldots D+1]\) as an array of \(D + 1\) zeros.
		}
		\Nbox{
			\For{\(j \in \{1 \dots n\}\)}
			\State \(i \gets A[j] - a_{min} + 1\) \Comment{ex. \(C[1]\) is the number of \(a_{min}\) seen in \(A\)}
			\State \(C[i] \gets C[i] + 1\)
			\EndFor
		}
		\LComment{Fill in the sorted array}
		\Nbox{\State Initialize \(S[1..n]\) as en empty array.}
		\State \(start \gets 1\)
		\Dbox{
			\For{\(i \in \{1, \dots D + 1\}\)}
			\State \(end \gets start + C[i]\) \Comment{find the contiguous blocks of \(S\)}
			\State \(d \gets a_{min} + i - 1\) \Comment{\(d\) is the value as seen in \(A\)}
			\State \Call{fill}{\(S[start \ldots end]\), \(d\)} \Comment{populate \(S\) with \(d\) in the given range}
			\State \(start \gets end + 1\)
			\EndFor
		}

		\State \Return \(S\)
	\end{algorithmic}
\end{algorithm}

We proceed to prove the runtime of the algorithm. The algorithm
first finds the minimum and maximum values in the array, which trivially is
\(O(n)\) time. Next, the algorithm initializes an array of size \(D + 1\)
with zeros, which takes \(O(D) \subset O(D+n)\) time. The algorithm then counts
the number of occurrences of each value in the array using \(O(n)\) time.
To examine the loop on line 11, see that \(C[i] = end_i - start_i\)
where \(end_i\) and \(start_i\) are the values of the variables at line 12 of the loop.
We have \(C[i]\) as the size of the contiguous block of \(S\).
Observe that
\(\sum_{i}C[i] = n\). Then, since each call to \textsc{fill} takes \(O(end - start)\) time,
the total running time of the work \textsc{fill} performs is
\begin{align*}
	\sum_{i}^{D+1} O(end_i - start_i) & = \sum_{i}^{D+1} O(C[i]) \\
	                                  & = O(n).                  \\
\end{align*}
Since there are \(D + 1\) iterations of the loop, the total running time of the loop
outside of the \textsc{fill} function is \(O(D)\). Therefore, the running time of the loop is \(O(n + D)\),
allowing us to conclude that the running time of the entire algorithm is \(O(D+n)\).

}

