\section{Problem 3}

You are interested in analyzing some remote data from two separate databases. 
Each database contains $n$ numerical values—so there are $2$n values total, and you may assume that no two values are the same. 
You'd like to determine the median of this set of $2n$ values, which we will define here to be the $n$th smallest value.
However, the only way you can access these values is through queries to the databases. 
In a single query, you can specify a value $k$ to one of the two databases, 
and the chosen database will return the $k$th smallest value that it contains. 
Since queries are expensive, you would like to compute the median using as few queries as possible.

\begin{enumerate}
\item Devise an algorithm that finds the median value using at most $O(\log n)$ queries. 
    Explain why your algorithm achieves this query bound.
\item Write a small program to test the method that you devised. 
    You may assume that the databases are just two sorted arrays and you want to make $O(\log n)$ accesses to those arrays. 
    Demonstrate it running on an example.
\end{enumerate}

\nanswer{1} 
\todo{
    Probably need to do the following:
    \begin{itemize}
        \item Describe the algorithm
        \item Prove correctness/termination
        \item Analyze the running time
    \end{itemize}
}

\nanswer{2}
\todo{
    Include code demonstration/figures
}